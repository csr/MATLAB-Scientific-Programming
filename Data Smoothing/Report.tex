\documentclass{article}

\usepackage{lipsum}
\usepackage[margin=1.2in]{geometry}
\usepackage{titlesec}
\usepackage{graphicx}
\usepackage{amsmath}

\newcommand{\code}{\texttt}
\newcommand{\norm}[1]{\left\lVert#1\right\rVert}

\usepackage{siunitx} % Required for alignment


% Specify images directory
\graphicspath{ {./report-images/} }

% Header and Footer stuff
\usepackage{fancyhdr}
\pagestyle{fancy}
\fancyhead{}
\fancyfoot{}
\fancyfoot[R]{ \thepage\ }
\renewcommand{\headrulewidth}{0pt}
\renewcommand{\footrulewidth}{0pt}
\newcommand{\sectionbreak}{\clearpage}
\setlength{\parindent}{0pt}

%

\begin{document}

%----------------------------------------------------------------------------------------
%	TITLE PAGE
%----------------------------------------------------------------------------------------

\begin{titlepage} % Suppresses displaying the page number on the title page and the subsequent page counts as page 1
	\newcommand{\HRule}{\rule{\linewidth}{0.5mm}}% Defines a new command for horizontal lines, change thickness here
	
	\center % Centre everything on the page
	
	%------------------------------------------------
	%	Headings
	%------------------------------------------------
	
	\textsc{\Large Data Smoothing}\\[0.5cm] % Major heading such as course name
	
	\textsc{\large Exercise 3}\\[0.5cm] % Minor heading such as course title
	
	%------------------------------------------------
	%	Title
	%------------------------------------------------
	
	\HRule\\[0.6cm]
	
	{\huge\bfseries Data Smoothing Report}\\[0.25cm] % Title of your document
	
	\HRule\\[1.5cm]
	
	%------------------------------------------------
	%	Author(s)
	%------------------------------------------------
	
	\begin{minipage}{0.4\textwidth}
		\begin{flushleft}
			\large
			\textit{Author}\\
			\textsc{Cesare De Cal} % Your name
		\end{flushleft}
	\end{minipage}
	~
	\begin{minipage}{0.4\textwidth}
		\begin{flushright}
			\large
			\textit{Professor}\\
			\textsc{Annie Cuyt}\\ % Supervisor's name
			[0.25cm]
			\textit{Assistant Professor}\\
			\textsc{Ferre Knaepkens} % Supervisor's name

		\end{flushright}
	\end{minipage}
		
	\vfill\vfill\vfill
	
	{\large\today}
		
	\vfill
	
\end{titlepage}

%----------------------------------------------------------------------------------------

\section{Introduction}\label{sec:intro}
This exercise asks to use the linearly independent basis functions:
$$\Phi_{3,i}(x)=$$
to find the optimal combination
$$\Phi(x)=\lambda_0(x)$$

that minimizes

for the 20 data points $(x_j,y_j)$ given in

\begin{table}[!ht]
\large        %% not "\fontsize{12}{12}\selectfont"
\centering    %% not "\center{...}"
\begin{tabular}{|c|c|c|}
\hline
$j$ & $x_j$ & $y_j$ \\     %% no "&" at start of row
\hline
0 & 0.0 & -0.80\\
1 & 0.6 & -0.34\\
2 & 1.5 & 0.59\\
3 & 1.7 & 0.59\\
4 & 1.9 & 0.23\\
5 & 2.1 & 0.10\\
6 & 2.3 & 0.28\\
7 & 2.6 & 1.03\\
8 & 2.8 & 1.50\\
9 & 3.0 & 1.44\\
10 & 3.6 & 0.74\\
11 & 4.7 & -0.82\\
12 & 5.2 & -1.27\\
13 & 5.7 & -0.92\\
14 & 5.8 & -0.92\\
15 & 6.0 & -1.04\\
16 & 6.4 & -0.79\\
17 & 6.9 & -0.06\\
18 & 7.6 & 1.00\\
19 & 8.0 & 0.00\\    
\hline
\end{tabular}
\end{table}

\section{Tools}
The following programming language and libraries have been used in this exercise:
\begin{itemize}
  \item Item 1
  \item C Math Library
  \item GSL (GNU Scientific Library)
\end{itemize}
The following double-precision GSL data types have been used in the exercise:
\begin{itemize}
  \item \code{gsl\_vector ?}
\end{itemize}
The following GSL methods have been used in the exercise:
\begin{itemize}
  \item \code{gsl\_matrix\_alloc(size1, size2)}
  \item \code{gsl\_matrix\_set\_zero(matrix)}
  \item \code{gsl\_matrix\_set(matrix, row, column, value)}
  \item \code{gsl\_matrix\_get(matrix, row, column)}
  \item \code{gsl\_vector\_alloc(size)}
  \item \code{gsl\_vector\_set\_zero(vector)}
  \item \code{gsl\_vector\_set(vector, index, value)}
  \item \code{gsl\_vector\_get(vector, index)}
  \item \code{gsl\_matrix\_memcpy(matrixToCopyFrom, matrix)}
  \item \code{gsl\_linalg\_SV\_decomp(A, V, S, workspaceVector)}
  \item \code{gsl\_vector\_minmax(vector, minInVector, maxInVector)}
\end{itemize}
In order to factorize a matrix into the LU decomposition, and then solve the square system $Ax=y$ using the decomposition of A, I've used the following methods:
\begin{itemize}
  \item \code{gsl\_linalg\_LU\_decomp(A, permutation, signum)}
  \item \code{gsl\_linalg\_LU\_solve(LU, permutation, b, x)}
  \item \code{gsl\_permutation\_alloc(size)}
\end{itemize}
The following method from the C Math library was used in this exercise to calculate the absolute value of a number:
\begin{itemize}
  \item \code{fabs(x)}
\end{itemize}
  
\section{Computation}


\section{Plot}
\section{Observations}
\end{document}