\documentclass{article}

\usepackage{lipsum}
\usepackage[margin=1.5in]{geometry}
\usepackage{titlesec}
\usepackage{graphicx}
\usepackage{amsmath}

\newcommand{\code}{\texttt}
\newcommand{\norm}[1]{\left\lVert#1\right\rVert}

\usepackage{siunitx} % Required for alignment


% Specify images directory
\graphicspath{ {./report-images/} }

% Header and Footer stuff
\usepackage{fancyhdr}
\pagestyle{fancy}
\fancyhead{}
\fancyfoot{}
\fancyfoot[R]{ \thepage\ }
\renewcommand{\headrulewidth}{0pt}
\renewcommand{\footrulewidth}{0pt}
\newcommand{\sectionbreak}{\clearpage}
\setlength{\parindent}{0pt}

%

\begin{document}

%----------------------------------------------------------------------------------------
%	TITLE PAGE
%----------------------------------------------------------------------------------------

\begin{titlepage} % Suppresses displaying the page number on the title page and the subsequent page counts as page 1
	\newcommand{\HRule}{\rule{\linewidth}{0.5mm}}% Defines a new command for horizontal lines, change thickness here
	
	\center % Centre everything on the page
	
	%------------------------------------------------
	%	Headings
	%------------------------------------------------
	
	\textsc{\Large Data Smoothing}\\[0.5cm] % Major heading such as course name
	
	\textsc{\large Exercise 3}\\[0.5cm] % Minor heading such as course title
	
	%------------------------------------------------
	%	Title
	%------------------------------------------------
	
	\HRule\\[0.6cm]
	
	{\huge\bfseries Data Smoothing Report}\\[0.25cm] % Title of your document
	
	\HRule\\[1.5cm]
	
	%------------------------------------------------
	%	Author(s)
	%------------------------------------------------
	
	\begin{minipage}{0.4\textwidth}
		\begin{flushleft}
			\large
			\textit{Author}\\
			\textsc{Cesare De Cal} % Your name
		\end{flushleft}
	\end{minipage}
	~
	\begin{minipage}{0.4\textwidth}
		\begin{flushright}
			\large
			\textit{Professor}\\
			\textsc{Annie Cuyt}\\ % Supervisor's name
			[0.25cm]
			\textit{Assistant Professor}\\
			\textsc{Ferre Knaepkens} % Supervisor's name

		\end{flushright}
	\end{minipage}
		
	\vfill\vfill\vfill
	
	{\large\today}
		
	\vfill
	
\end{titlepage}

%----------------------------------------------------------------------------------------

\section{Introduction}\label{sec:intro}
This exercise asks to use the linearly independent basis functions:
$$\Phi_{3,i}(x)=$$
to find the optimal combination
$$\Phi(x)=\lambda_0(x)$$

that minimizes

for the 20 data points $(x_j,y_j)$ given in

\begin{table}[!ht]
\large        %% not "\fontsize{12}{12}\selectfont"
\centering    %% not "\center{...}"
\begin{tabular}{|c|c|c|}
\hline
$j$ & $x_j$ & $y_j$ \\     %% no "&" at start of row
\hline
0 & 0.0 & -0.80\\
1 & 0.6 & -0.34\\
2 & 1.5 & 0.59\\
3 & 1.7 & 0.59\\
4 & 1.9 & 0.23\\
5 & 2.1 & 0.10\\
6 & 2.3 & 0.28\\
7 & 2.6 & 1.03\\
8 & 2.8 & 1.50\\
9 & 3.0 & 1.44\\
10 & 3.6 & 0.74\\
11 & 4.7 & -0.82\\
12 & 5.2 & -1.27\\
13 & 5.7 & -0.92\\
14 & 5.8 & -0.92\\
15 & 6.0 & -1.04\\
16 & 6.4 & -0.79\\
17 & 6.9 & -0.06\\
18 & 7.6 & 1.00\\
19 & 8.0 & 0.00\\    
\hline
\end{tabular}
\end{table}

\section{Tools}
The following programming language and libraries have been used in this exercise:
\begin{itemize}
  \item Item 1
  \item C Math Library
  \item GSL (GNU Scientific Library)
\end{itemize}
The following double-precision GSL data types have been used in the exercise:
\begin{itemize}
  \item \code{gsl\_matrix}
  \item \code{gsl\_vector}
\end{itemize}
The following GSL methods have been used in the exercise:
\begin{itemize}
  \item \code{gsl\_matrix\_alloc(size1, size2)}
  \item \code{gsl\_matrix\_set(matrix, row, column, value)}
  \item \code{gsl\_vector\_alloc(size)}
  \item \code{gsl\_vector\_set(vector, index, value)}
  \item \code{gsl\_vector\_get(vector, index)}
  \item \code{gsl\_linalg\_SV\_decomp(A, V, S, workspaceVector)}
  \item \code{gsl\_vector\_minmax(vector, minInVector, maxInVector)}
\end{itemize}
In order to factorize a matrix into the QR decomposition, and then solve the square system $Ax=y$ using the decomposition of A, I've used the following methods:
\begin{itemize}
  \item \code{gsl\_linalg\_LU\_decomp(A, permutation, signum)}
  \item \code{gsl\_linalg\_LU\_solve(LU, permutation, b, x)}
  \item \code{gsl\_permutation\_alloc(size)}
\end{itemize}
The following method from the C Math library was used in this exercise to calculate the absolute value of a number:
\begin{itemize}
  \item \code{fabs(x)}
\end{itemize}
  
\section{Computation}
First off, I compute the coefficients A of the linear system by using the linearly independent basis function. This is what A looks like:

$$
\begin{bmatrix} 
1.000000000000000e+00 & 0.000000000000000e+00 & 0.000000000000000e+00 & 0.000000000000000e+00 \\
6.400000000000002e-02 & 2.880000000000000e-01 & 4.320000000000001e-01 & 2.160000000000000e-01 \\
-1.250000000000000e-01 & 1.125000000000000e+00 & -3.375000000000000e+00 & 3.375000000000000e+00 \\
-3.429999999999999e-01 & 2.499000000000000e+00 & -6.068999999999998e+00 & 4.912999999999999e+00 \\
-7.289999999999998e-01 & 4.616999999999998e+00 & -9.747000000000000e+00 & 6.858999999999999e+00 \\
-1.331000000000000e+00 & 7.623000000000002e+00 & -1.455300000000000e+01 & 9.261000000000001e+00 \\
-2.196999999999999e+00 & 1.166100000000000e+01 & -2.063099999999999e+01 & 1.216700000000000e+01 \\
-4.096000000000001e+00 & 1.996800000000001e+01 & -3.244800000000000e+01 & 1.757600000000000e+01 \\
-5.831999999999998e+00 & 2.721599999999999e+01 & -4.233599999999999e+01 & 2.195199999999999e+01 \\
-8.000000000000000e+00 & 3.600000000000000e+01 & -5.400000000000000e+01 & 2.700000000000000e+01 \\
-1.757600000000000e+01 & 7.300800000000001e+01 & -1.010880000000000e+02 & 4.665600000000001e+01 \\
-5.065300000000001e+01 & 1.930290000000000e+02 & -2.451990000000000e+02 & 1.038230000000000e+02 \\
-7.408800000000001e+01 & 2.751840000000000e+02 & -3.407040000000000e+02 & 1.406080000000000e+02 \\
-1.038230000000000e+02 & 3.777390000000001e+02 & -4.581090000000000e+02 & 1.851930000000000e+02 \\
-1.105920000000000e+02 & 4.008960000000000e+02 & -4.844160000000000e+02 & 1.951120000000000e+02 \\
-1.250000000000000e+02 & 4.500000000000000e+02 & -5.400000000000000e+02 & 2.160000000000000e+02 \\
-1.574640000000000e+02 & 5.598720000000002e+02 & -6.635520000000001e+02 & 2.621440000000001e+02 \\
-2.053790000000000e+02 & 7.205670000000001e+02 & -8.426970000000001e+02 & 3.285090000000001e+02 \\
-2.874960000000000e+02 & 9.931679999999998e+02 & -1.143648000000000e+03 & 4.389759999999999e+02 \\
-3.430000000000000e+02 & 1.176000000000000e+03 & -1.344000000000000e+03 & 5.120000000000000e+02 \\\end{bmatrix}
$$

The column vector $\vec{b}$ is formed by the input $y$ values:

$$
\begin{bmatrix} 
-8.000000000000000e-01\\
-3.400000000000000e-01\\
5.900000000000000e-01\\
5.900000000000000e-01\\
2.300000000000000e-01\\
1.000000000000000e-01\\
2.800000000000000e-01\\
1.030000000000000e+00\\
1.500000000000000e+00\\
1.440000000000000e+00\\
7.400000000000000e-01\\
-8.200000000000000e-01\\
-1.270000000000000e+00\\
-9.200000000000000e-01\\
-9.200000000000000e-01\\
-1.040000000000000e+00\\
-7.900000000000000e-01\\
-6.000000000000000e-02\\
1.000000000000000e+00\\
0.000000000000000e+00\\
\end{bmatrix}
$$

In order to calculate the residual of this system, I first need to calculate the missing $\lambda$ vector. Since this is an overdetermined system, I can use QR decomposition to solve the system. QR decomposition of A:

$$
\begin{bmatrix} 
-5.607137694902810e+02 & 1.956202843752372e+03 & -2.276212763756861e+03 & 8.833928944999552e+02 \\
1.139370324819986e-04 & -4.671135706796601e+01 & 1.122030535262658e+02 & -6.782418228480103e+01 \\
-2.225332665664034e-04 & 1.460497799957120e-02 & 7.581872706318327e+00 & -8.735870723873337e+00 \\
-6.106312834582108e-04 & 2.762429988027977e-02 & 1.797466673315370e-01 & -2.470454024049311e+00 \\
-1.297814010615264e-03 & 4.400918925292650e-02 & 2.113118885654055e-01 & 1.354991299702775e-01 \\
-2.369534222399064e-03 & 6.326913656703972e-02 & 2.350053228690535e-01 & 9.686987975376644e-02 \\
-3.911244693171105e-03 & 8.491363227214763e-02 & 2.513310790520081e-01 & 6.050043190845661e-02 \\
-7.291970078847909e-03 & 1.207784144212678e-01 & 2.631080548709738e-01 & 1.049772432830278e-02 \\
-1.038251208492211e-02 & 1.462383018205487e-01 & 2.632200917938682e-01 & -1.959354810540480e-02 \\
-1.424212906024982e-02 & 1.723659541841733e-01 & 2.577288314294048e-01 & -4.692208258102275e-02 \\
-3.128995754536886e-02 & 2.498504055563917e-01 & 2.126763547071380e-01 & -1.108942994397425e-01 \\
-9.017582041110428e-02 & 3.520968788593226e-01 & 5.605686650054093e-02 & -1.473630562460556e-01 \\
-1.318963572269736e-01 & 3.635568863958127e-01 & -2.831753402090372e-02 & -1.240982343374392e-01 \\
-1.848325706777897e-01 & 3.424022086740725e-01 & -1.099043818739413e-01 & -7.278910904340619e-02 \\
-1.968831921288935e-01 & 3.335830602668363e-01 & -1.251940955112117e-01 & -5.896438574071358e-02 \\
-2.225332665664034e-01 & 3.108660410390103e-01 & -1.542417580429807e-01 & -2.760834500928011e-02 \\
-2.803278262928973e-01 & 2.434003295032724e-01 & -2.044456433018340e-01 & 5.043304854412397e-02 \\
-3.656292780331310e-01 & 1.127004233192019e-01 & -2.472054281433762e-01 & 1.782091144383195e-01 \\
-5.118193920381977e-01 & -1.705668773785795e-01 & -2.556299463550832e-01 & 4.175378294298211e-01 \\
-6.106312834582110e-01 & -3.921595584433181e-01 & -2.261060282111244e-01 & 5.880648466528077e-01 \\
\end{bmatrix}
$$

The following are the $\lambda$ found:

$$
\begin{bmatrix}
-1.118262027055321e+00\\
-4.515724133992401e-01\\
2.972563550105292e-03\\
2.972882574075207e-01\\
\end{bmatrix}
$$

The condition number of the matrix $A$ is equal to $3.741019262503867e+03$.

\section{Plot}
\section{Observations}
\end{document}