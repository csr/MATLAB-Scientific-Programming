\documentclass{article}

\usepackage{lipsum}
\usepackage[margin=1.2in]{geometry}
\usepackage{titlesec}
\usepackage{graphicx}
\usepackage{amsmath}

\newcommand{\code}{\texttt}
\newcommand{\norm}[1]{\left\lVert#1\right\rVert}

\usepackage{siunitx} % Required for alignment

\sisetup{
  round-mode          = places, % Rounds numbers
  round-precision     = 2, % to 2 places
}

% Specify images directory
\graphicspath{ {./report-images/} }

% Header and Footer stuff
\usepackage{fancyhdr}
\pagestyle{fancy}
\fancyhead{}
\fancyfoot{}
\fancyfoot[R]{ \thepage\ }
\renewcommand{\headrulewidth}{0pt}
\renewcommand{\footrulewidth}{0pt}
\newcommand{\sectionbreak}{\clearpage}
\setlength{\parindent}{0pt}

%

\begin{document}

%----------------------------------------------------------------------------------------
%	TITLE PAGE
%----------------------------------------------------------------------------------------

\begin{titlepage} % Suppresses displaying the page number on the title page and the subsequent page counts as page 1
	\newcommand{\HRule}{\rule{\linewidth}{0.5mm}}% Defines a new command for horizontal lines, change thickness here
	
	\center % Centre everything on the page
	
	%------------------------------------------------
	%	Headings
	%------------------------------------------------
	
	\textsc{\Large Systems of Linear Equations}\\[0.5cm] % Major heading such as course name
	
	\textsc{\large Exercise 7}\\[0.5cm] % Minor heading such as course title
	
	%------------------------------------------------
	%	Title
	%------------------------------------------------
	
	\HRule\\[0.6cm]
	
	{\huge\bfseries Solving a Linear System with LU Decomposition}\\[0.25cm] % Title of your document
	
	\HRule\\[1.5cm]
	
	%------------------------------------------------
	%	Author(s)
	%------------------------------------------------
	
	\begin{minipage}{0.4\textwidth}
		\begin{flushleft}
			\large
			\textit{Author}\\
			\textsc{Cesare De Cal} % Your name
		\end{flushleft}
	\end{minipage}
	~
	\begin{minipage}{0.4\textwidth}
		\begin{flushright}
			\large
			\textit{Professor}\\
			\textsc{Annie Cuyt}\\ % Supervisor's name
			[0.25cm]
			\textit{Assistant Professor}\\
			\textsc{Ferre Knaepkens} % Supervisor's name

		\end{flushright}
	\end{minipage}
		
	\vfill\vfill\vfill
	
	{\large\today}
		
	\vfill
	
\end{titlepage}

%----------------------------------------------------------------------------------------

\section{Introduction}\label{sec:intro}
In this exercise I am going to use solve a linear system $A\vec{x}=\vec{y}$ using LU decomposition. The goal is to verify that first element of the unknowns vector, $x_{1}$, will contain an approximation of $e - 2$.\\

As we've seen in class, there are multiple ways of solving a linear system $AX=B$. Assume $A$ is a $n\times n$ square matrix, $B$ is a "constant" term matrix $n\times h$, and $X$ is a $n\times h$ unknown matrix. To solve for $X$, we could compute the inverse of $A$ and find $x=A^{-1}y$. We've seen that this approach, however, requires more computations than necessary and returns a less accurate result.\\

On the other hand, the LU decomposition technique is a way to represent the matrix A in the form of simpler matrices, $L$ and $U$ (lower triangular and upper triangular matrices, respectively):
$$PA=LU$$

This method uses forward substitution (solving for $Y$ from $LY=B$) and backward substitution (solving for $X$ from $UX = Y$). I'll be specifically solving the system by using Gaussian Elimination with partial pivoting, which reduces round-offs errors compared to its naive implementation. I'll also be calculating the error and the condition number as the variable $n$ increases, and plot the results.

\section{Tools}
The following programming language and libraries have been used in this exercise:
\begin{itemize}
  \item C
  \item GSL (GNU Scientific Library)
\end{itemize}
The following double-precision GSL data types have been used in the exercise:
\begin{itemize}
  \item \code{gsl\_vector}
  \item \code{gsl\_matrix}
  \item \code{gsl\_permutation}
\end{itemize}
The following GSL methods have been used in the exercise:
\begin{itemize}
  \item \code{gsl\_matrix\_alloc(size1, size2)}
  \item \code{gsl\_matrix\_set\_zero(matrix)}
  \item \code{gsl\_matrix\_set(matrix, row, column, value)}
  \item \code{gsl\_matrix\_get(matrix, row, column)}
  \item \code{gsl\_vector\_alloc(size)}
  \item \code{gsl\_vector\_set\_zero(vector)}
  \item \code{gsl\_vector\_set(vector, index, value)}
  \item \code{gsl\_vector\_get(vector, index)}
  \item \code{gsl\_matrix\_memcpy(matrixToCopyFrom, matrix)}
  \item \code{gsl\_linalg\_SV\_decomp(A, V, S, workspaceVector)}
  \item \code{gsl\_vector\_minmax(vector, minInVector, maxInVector)}
\end{itemize}
In order to factorize a matrix into the LU decomposition, and then solve the square system $Ax=y$ using the decomposition of A, I've used the following methods:
\begin{itemize}
  \item \code{gsl\_linalg\_LU\_decomp(A, permutation, signum)}
  \item \code{gsl\_linalg\_LU\_solve(LU, permutation, b, x)}
  \item \code{gsl\_permutation\_alloc(size)}
\end{itemize}
  
\section{Solving the linear system}
In order to solve the system $Ax=y$, I first need to build the matrix A by understanding how it's build. The requirements are to build a tridiagonal matrix with the values $-1$ on the adjacent upper diagonal, the entries $+1$ on the adjacent lower diagonal, and on the main diagonal the values $b_{i}$, with $i = 1, \ldots, n$ given by

$$b_i =\frac{2(i+1)}{3},\quad i + 1= 3, 6, 9,\ldots$$
$$b_i =1,\quad i + 1 = 2, 4, 5, 7, 8, \ldots$$

By looking closely at the first rule, we see that the $i+1$ are all multiples of 3 ($i+1 = 3*k$, for some $k$). Hence the $i$ are of the form $i = 3*k-1$, for some $k$. For $n = 5$, for example, this is what the matrix looks like:
$$
\begin{bmatrix}
1.0000000000 & -1.0000000000 & 0.0000000000 & 0.0000000000 & 0.0000000000 \\
1.0000000000 & 2.0000000000 & -1.0000000000 & 0.0000000000 & 0.0000000000 \\
0.0000000000 & 1.0000000000 & 1.0000000000 & -1.0000000000 & 0.0000000000 \\ 
0.0000000000 & 0.0000000000 & 1.0000000000 & 1.0000000000 & -1.0000000000 \\ 
0.0000000000 & 0.0000000000 & 0.0000000000 & 1.0000000000 & 4.0000000000 \\\end{bmatrix}
$$

The coefficients matrix A is first allocated by using the \code{gsl\_matrix\_alloc} method, then I set all the elements to zero with \code{gsl\_matrix\_set\_zero} and finally nested \code{for} loops fill the diagonal values by checking the indexes. The coefficients reported above on the diagonal have 5 significant digits for improve the readability of this report.\\

I used the \code{gsl\_vector\_alloc} method to create an instance of the vector. All of its elements were set to zero by using \code{gsl\_vector\_set\_zero(vector)}. The exercise asks us to set the first element of the $y$ vector to one, so I used \code{gsl\_vector\_set(vector, 0, 1)} to assign the value 1 to index 0. For $n=5$, we have:
$$
\vec{y}=
\begin{bmatrix}
1.0000000000 \\
0.0000000000 \\
0.0000000000 \\
0.0000000000 \\
0.0000000000 \\
\end{bmatrix}
$$

Given the $Ax=y$ system, my goal is now to find the vector of the unknowns $x$. To do so, I first factorize $A$ into its LU decomposition by allocating a new matrix (so that the matrix which represents $A$ doesn't get overridden) using \code{gsl\_matrix\_memcpy} and then by calling \code{gsl\_linalg\_LU\_decomp}. This method utilizes Gaussian Elimination with partial pivoting to compute the decomposition. The following is the $LU$ matrix for $n=5$:
$$
LU=
\begin{bmatrix} 
1.0000000000 & -1.0000000000 & 0.0000000000 & 0.0000000000 & 0.0000000000\\ 
1.0000000000 & 3.0000000000 & -1.0000000000 & 0.0000000000 & 0.0000000000\\ 
0.0000000000 & 0.3333333333 & 1.3333333333 & -1.0000000000 & 0.0000000000\\ 
0.0000000000 & 0.0000000000 & 0.7500000000 & 1.7500000000 & -1.0000000000\\ 
0.0000000000 & 0.0000000000 & 0.0000000000 & 0.5714285714 & 4.5714285714\\
\end{bmatrix}
$$

I can now use the $LU$ matrix to solve the system by passing $LU$, $x$, a permutation structure \code{gsl\_permutation} and $y$ to \code{gsl\_linalg\_LU\_solve}. This method uses forward and back-substitution to modify the contents of the $x$ vector given in input, which now looks like this (for $n=5$):

$$
\vec{x}=
\begin{bmatrix}
0.7187500000\\
-0.2812500000\\
0.1562500000\\
-0.1250000000\\
0.0312500000\\
\end{bmatrix}
$$

The first element looks contains an approximation of $e-2$. By increasing the size of the matrix $n$, $x_i$ becomes increasingly more precise. For $n=10$, for example:

$$
\vec{x}=
\begin{bmatrix}
0.7182817183\\
-0.2817182817\\
0.1548451548\\
-0.1268731269\\
0.0279720280\\
-0.0149850150\\
0.0129870130\\
-0.0019980020\\
0.0009990010\\
-0.0009990010\\
\end{bmatrix}
$$

Then, I calculate the condition number of the matrix $A$ of order $n$ which will give me a better idea if this is a well-conditioned or an ill-conditioned linear system. In GSL there is no direct function that calculates the condition number, but it's possible to use the ratio of the largest singular value of matrix A, $\sigma_n (A)$, to the smallest $\sigma_1 (A)$:

$$\kappa(A) := \frac{\sigma_n (A)}{\sigma_1 (A)}= \frac{\norm{A}}{\norm{A^{-1}}^{-1}}$$

I proceed to factorize $A$ into its singular value decomposition $SVD$ using the \code{gsl\_linalg\_SV\_decomp} method, and then use $\code{gsl\_vector\_minmax}$ to extract the minimum and maximum singular values out of the vector $S$ that contains the diagonal elements of the singular value matrix. \\

For $n=5$, the condition number is

$$\kappa(A) = \frac{\sigma_n (A)}{\sigma_1 (A)}= \frac{4.2051006107}{1.1426432872}=3.6801516779$$

For $n=10$, the condition number is

$$\kappa(A) = \frac{\sigma_n (A)}{\sigma_1 (A)}= \frac{6.2820508697}{1.1424251953}=5.4988728328$$

I calculate the error by subtracting the computed solution $x_{1}^{\ast}$ from the exact mathematical solution $\widetilde{x}$ (which can be obtained by using the $\code{M\_E}$ GSL constant minus 2).\\

Finally, I am now going to insert the $x_1$ component, the error, and the condition number for $n$ from 1 to 50 in the next page. I chose $n = 50$ as the upper limit because it looks like the error doesn't get better after $n=20$.
\begin{table*}[htb]
\centering % used for centering table
\begin{tabular}{c c c c} % centered columns (4 columns)
$n$ & $\widetilde{x}_1$ & $x_1^{\ast}- \widetilde{x}_1$ & $\kappa(A_n)$ \\ [0.65ex] % inserts table
%heading
\hline % inserts single horizontal line
1 & 1.0000000000000000 & -0.2817181715409549 & 1.0000000000000000 \\
2 & 0.6666666666666667 & 0.0516151617923784 & 1.7675918792439982 \\
3 & 0.7500000000000000 & -0.0317181715409549 & 2.5615528128088298 \\
4 & 0.7142857142857142 & 0.0039961141733309 & 2.2586960380558874 \\
5 & 0.7187500000000000 & -0.0004681715409549 & 3.6801516778795333 \\
6 & 0.7179487179487180 & 0.0003331105103271 & 3.9538640020225495 \\
7 & 0.7183098591549295 & -0.0000280306958844 & 3.8476746091189153 \\
8 & 0.7182795698924731 & 0.0000022585665720 & 5.3770375880477213 \\
9 & 0.7182835820895522 & -0.0000017536305071 & 5.7275818392893347 \\
10 & 0.7182817182817183 & 0.0000001101773268 & 5.4988728328025962 \\
11 & 0.7182818352059925 & -0.0000000067469474 & 7.1003357703679963 \\
12 & 0.7182818229439497 & 0.0000000055150954 & 7.5821646375997114 \\
13 & 0.7182818287356958 & -0.0000000002766507 & 7.1955317021216594 \\
14 & 0.7182818284454013 & 0.0000000000136438 & 8.8331498923754399 \\
15 & 0.7182818284705836 & -0.0000000000115385 & 9.4880747300490409 \\
16 & 0.7182818284585635 & 0.0000000000004816 & 8.9115586964085285 \\
17 & 0.7182818284590651 & -0.0000000000000200 & 10.5715228483523767 \\
18 & 0.7182818284590280 & 0.0000000000000171 & 11.4201824603811897 \\
19 & 0.7182818284590458 & -0.0000000000000007 & 10.6381340746273825 \\
20 & 0.7182818284590452 & -0.0000000000000001 & 12.3131996586543000 \\
21 & 0.7182818284590453 & -0.0000000000000002 & 13.3688310431707524 \\
22 & 0.7182818284590453 & -0.0000000000000002 & 12.3710782126140035 \\
23 & 0.7182818284590453 & -0.0000000000000002 & 14.0570047940165566 \\
24 & 0.7182818284590453 & -0.0000000000000002 & 15.3286298262578065 \\
25 & 0.7182818284590453 & -0.0000000000000002 & 14.1081637697318296 \\
26 & 0.7182818284590453 & -0.0000000000000002 & 15.8022624874376820 \\
27 & 0.7182818284590453 & -0.0000000000000002 & 17.2963070591942554 \\
28 & 0.7182818284590453 & -0.0000000000000002 & 15.8480934751923019 \\
29 & 0.7182818284590453 & -0.0000000000000002 & 17.5485561661636069 \\
30 & 0.7182818284590453 & -0.0000000000000002 & 19.2697572434997504 \\
31 & 0.7182818284590453 & -0.0000000000000002 & 17.5900604343512654 \\
32 & 0.7182818284590453 & -0.0000000000000002 & 19.2956148454130130 \\
33 & 0.7182818284590453 & -0.0000000000000002 & 21.2475632526110978 \\
34 & 0.7182818284590453 & -0.0000000000000002 & 19.3335364470429454 \\
35 & 0.7182818284590453 & -0.0000000000000002 & 21.0432545577449801 \\
36 & 0.7182818284590453 & -0.0000000000000002 & 23.2287362210777601 \\
37 & 0.7182818284590453 & -0.0000000000000002 & 21.0781612793335533 \\
38 & 0.7182818284590453 & -0.0000000000000002 & 22.7913459874268973 \\
39 & 0.7182818284590453 & -0.0000000000000002 & 25.2125651999996911 \\
40 & 0.7182818284590453 & -0.0000000000000002 & 22.8236808378170366 \\
41 & 0.7182818284590453 & -0.0000000000000002 & 24.5397955641461145 \\
42 & 0.7182818284590453 & -0.0000000000000002 & 27.1985260013623389 \\
43 & 0.7182818284590453 & -0.0000000000000002 & 24.5699107748932342 \\
44 & 0.7182818284590453 & -0.0000000000000002 & 26.2885339032351872 \\
45 & 0.7182818284590453 & -0.0000000000000002 & 29.1862237020870907 \\
46 & 0.7182818284590453 & -0.0000000000000002 & 26.3167141046176667 \\
47 & 0.7182818284590453 & -0.0000000000000002 & 28.0375084620521520 \\
48 & 0.7182818284590453 & -0.0000000000000002 & 31.1753551463410439 \\
49 & 0.7182818284590453 & -0.0000000000000002 & 28.0639869168957361 \\
50 & 0.7182818284590453 & -0.0000000000000002 & 29.7866787202101015 \\
\hline %inserts single line
\end{tabular}
\end{table*}

\section{Observations}
It can be observed that as $n$ increases, the $\widetilde{x}_1$ component gets incrementally closer to the actual $e-2$ value. The condition number also gets incrementally bigger. It can be noticed, however, that a large condition number doesn’t necessarily mean that the error will be large in all cases, just that it is possible to have a large error.\\

The linear system appears to be ill-conditioned for most $n$.

\end{document}