\documentclass{article}

\usepackage{lipsum}
\usepackage[margin=1.2in]{geometry}
\usepackage{titlesec}
\usepackage{graphicx}
\usepackage{amsmath}

\newcommand{\code}{\texttt}

% Specify images directory
\graphicspath{ {./report-images/} }

% Header and Footer stuff
\usepackage{fancyhdr}
\pagestyle{fancy}
\fancyhead{}
\fancyfoot{}
\fancyfoot[R]{ \thepage\ }
\renewcommand{\headrulewidth}{0pt}
\renewcommand{\footrulewidth}{0pt}
\newcommand{\sectionbreak}{\clearpage}
\setlength{\parindent}{0pt}

%

\begin{document}

%----------------------------------------------------------------------------------------
%	TITLE PAGE
%----------------------------------------------------------------------------------------

\begin{titlepage} % Suppresses displaying the page number on the title page and the subsequent page counts as page 1
	\newcommand{\HRule}{\rule{\linewidth}{0.5mm}}% Defines a new command for horizontal lines, change thickness here
	
	\center % Centre everything on the page
	
	%------------------------------------------------
	%	Headings
	%------------------------------------------------
	
	\textsc{\Large Systems of Linear Equations}\\[0.5cm] % Major heading such as course name
	
	\textsc{\large Exercise 7}\\[0.5cm] % Minor heading such as course title
	
	%------------------------------------------------
	%	Title
	%------------------------------------------------
	
	\HRule\\[0.6cm]
	
	{\huge\bfseries Systems of Linear Equations Report}\\[0.25cm] % Title of your document
	
	\HRule\\[1.5cm]
	
	%------------------------------------------------
	%	Author(s)
	%------------------------------------------------
	
	\begin{minipage}{0.4\textwidth}
		\begin{flushleft}
			\large
			\textit{Author}\\
			\textsc{Cesare De Cal} % Your name
		\end{flushleft}
	\end{minipage}
	~
	\begin{minipage}{0.4\textwidth}
		\begin{flushright}
			\large
			\textit{Professor}\\
			\textsc{Annie Cuyt}\\ % Supervisor's name
			[0.25cm]
			\textit{Assistant Professor}\\
			\textsc{Ferre Knaepkens} % Supervisor's name

		\end{flushright}
	\end{minipage}
		
	\vfill\vfill\vfill
	
	{\large\today}
		
	\vfill
	
\end{titlepage}

%----------------------------------------------------------------------------------------

\section{Introduction}\label{sec:intro}
The purpose of this exercise is to use solve a linear system using LU decomposition.\\

\section{Tools}
The following programming language and libraries have been used in this exercise:
\begin{itemize}
  \item C
  \item GSL (GNU Scientific Library)
\end{itemize}
The following GSL data types have been used in the exercise:
\begin{itemize}
  \item \code{gsl\_vector}
  \item \code{gsl\_matrix}
  \item \code{gsl\_permutation}
\end{itemize}
The following GSL methods have been used in the exercise:
\begin{itemize}
  \item \code{gsl\_matrix\_alloc(size1, size2)}
  \item \code{gsl\_matrix\_set\_zero(matrix)}
  \item \code{gsl\_matrix\_set(matrix, row, column, value)}
  \item \code{gsl\_matrix\_get(matrix, row, column)}
  \item \code{gsl\_vector\_alloc(size)}
  \item \code{gsl\_vector\_set\_zero(vector)}
  \item \code{gsl\_vector\_set(size)}
  \item \code{gsl\_vector\_get(vector, index)}
  \item \code{gsl\_permutation\_alloc(size)}
\end{itemize}
In order to factorize a matrix into the LU decomposition, and then solve the square system $Ax=b$ using the decomposition of A, I've used the following methods:
\begin{itemize}
  \item \code{gsl\_linalg\_LU\_decomp(A, permutation, signum)}
  \item \code{gsl\_linalg\_LU\_solve(LU, permutation, b, x)}
\end{itemize}
  
\section{Solving the system}
In order to solve the system, I first need to build the matrix A by understanding how it's build. The requirements are to build a tridiagonal matrix with the values $-1$ on the adjacent upper diagonal, the entries $+1$ on the adjacent lower diagonal, and on the main diagonal the values $b\_i$, with $i = 1, \ldots, n$ given by

$$b_i =\frac{2(i+1)}{3},\quad i + 1= 3, 6, 9,\ldots$$
$$b_i =1,\quad i + 1 = 2, 4, 5, 7, 8, \ldots$$

By looking closely at the first rule, we see that the $i+1$ are all multiples of 3 ($i+1 = 3*k$, for some $k$). Hence the $i$ are of the form $i = 3*k-1$, for some $k$. For $n = 10$, for example, this is what the matrix approximately looks like:

$$
\quad
\begin{bmatrix}
1.0000 & -1.0000 & 0.0000 & 0.0000 & 0.0000 & 0.0000 & 0.0000 & 0.0000 & 0.0000 & 0.0000 \\
1.0000 & 2.0000 & -1.0000 & 0.0000 & 0.0000 & 0.0000 & 0.0000 & 0.0000 & 0.0000 & 0.0000 \\
0.0000 & 1.0000 & 1.0000 & -1.0000 & 0.0000 & 0.0000 & 0.0000 & 0.0000 & 0.0000 & 0.0000 \\
0.0000 & 0.0000 & 1.0000 & 1.0000 & -1.0000 & 0.0000 & 0.0000 & 0.0000 & 0.0000 & 0.0000 \\
0.0000 & 0.0000 & 0.0000 & 1.0000 & 4.0000 & -1.0000 & 0.0000 & 0.0000 & 0.0000 & 0.0000 \\
0.0000 & 0.0000 & 0.0000 & 0.0000 & 1.0000 & 1.0000 & -1.0000 & 0.0000 & 0.0000 & 0.0000 \\
0.0000 & 0.0000 & 0.0000 & 0.0000 & 0.0000 & 1.0000 & 1.0000 & -1.0000 & 0.0000 & 0.0000 \\
0.0000 & 0.0000 & 0.0000 & 0.0000 & 0.0000 & 0.0000 & 1.0000 & 6.0000 & -1.0000 & 0.0000 \\
0.0000 & 0.0000 & 0.0000 & 0.0000 & 0.0000 & 0.0000 & 0.0000 & 1.0000 & 1.0000 & -1.0000 \\
0.0000 & 0.0000 & 0.0000 & 0.0000 & 0.0000 & 0.0000 & 0.0000 & 0.0000 & 1.0000 & 1.0000
\end{bmatrix}
$$

\section{Observations}

\end{document}