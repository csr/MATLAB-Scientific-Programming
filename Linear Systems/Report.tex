\documentclass{article}

\usepackage{lipsum}
\usepackage[margin=1.2in]{geometry}
\usepackage{titlesec}
\usepackage{graphicx}
\usepackage{amsmath}

\newcommand{\code}{\texttt}

% Specify images directory
\graphicspath{ {./report-images/} }

% Header and Footer stuff
\usepackage{fancyhdr}
\pagestyle{fancy}
\fancyhead{}
\fancyfoot{}
\fancyfoot[R]{ \thepage\ }
\renewcommand{\headrulewidth}{0pt}
\renewcommand{\footrulewidth}{0pt}
\newcommand{\sectionbreak}{\clearpage}
\setlength{\parindent}{0pt}

%

\begin{document}

%----------------------------------------------------------------------------------------
%	TITLE PAGE
%----------------------------------------------------------------------------------------

\begin{titlepage} % Suppresses displaying the page number on the title page and the subsequent page counts as page 1
	\newcommand{\HRule}{\rule{\linewidth}{0.5mm}}% Defines a new command for horizontal lines, change thickness here
	
	\center % Centre everything on the page
	
	%------------------------------------------------
	%	Headings
	%------------------------------------------------
	
	\textsc{\Large Systems of Linear Equations}\\[0.5cm] % Major heading such as course name
	
	\textsc{\large Exercise 7}\\[0.5cm] % Minor heading such as course title
	
	%------------------------------------------------
	%	Title
	%------------------------------------------------
	
	\HRule\\[0.6cm]
	
	{\huge\bfseries Systems of Linear Equations Report}\\[0.25cm] % Title of your document
	
	\HRule\\[1.5cm]
	
	%------------------------------------------------
	%	Author(s)
	%------------------------------------------------
	
	\begin{minipage}{0.4\textwidth}
		\begin{flushleft}
			\large
			\textit{Author}\\
			\textsc{Cesare De Cal} % Your name
		\end{flushleft}
	\end{minipage}
	~
	\begin{minipage}{0.4\textwidth}
		\begin{flushright}
			\large
			\textit{Professor}\\
			\textsc{Annie Cuyt}\\ % Supervisor's name
			[0.25cm]
			\textit{Assistant Professor}\\
			\textsc{Ferre Knaepkens} % Supervisor's name

		\end{flushright}
	\end{minipage}
		
	\vfill\vfill\vfill
	
	{\large\today}
		
	\vfill
	
\end{titlepage}

%----------------------------------------------------------------------------------------

\section{Introduction}\label{sec:intro}
In this exercise I am going to use solve a linear system $A\vec{x}=\vec{y}$ using LU decomposition. The goal is to verify that first element of the unknowns vector, $x_{1}$, will contain an approximation of $e - 2$.\\

As we've seen in class, there are multiple ways of solving a linear system $AX=B$. Assume $A$ is a $n\times n$ square matrix, $B$ is a "constant" term matrix $n\times h$, and $X$ is a $n\times h$ unknown matrix. To solve for $X$, we could compute the inverse of $A$ and find $x=A^{-1}y$. We've seen that this approach, however, requires more computations than necessary and returns a less accurate result.\\

On the other hand, the LU decomposition technique is a way to represent the matrix A in the form of simpler matrices, $L$ and $U$ (lower triangular and upper triangular matrices, respectively):
$$PA=LU$$

This method uses forward substitution (solving for $Y$ from $LY=B$) and backward substitution (solving for $X$ from $UX = Y$). I'll be specifically be solving the system by using Gaussian Elimination {\it with partial pivoting}, which reduces round-offs errors compared to its naive implementation, and then provide observations on the accuracy of the results obtained.

\section{Tools}
The following programming language and libraries have been used in this exercise:
\begin{itemize}
  \item C
  \item GSL (GNU Scientific Library)
\end{itemize}
The following GSL data types have been used in the exercise:
\begin{itemize}
  \item \code{gsl\_vector}
  \item \code{gsl\_matrix}
  \item \code{gsl\_permutation}
\end{itemize}
The following GSL methods have been used in the exercise:
\begin{itemize}
  \item \code{gsl\_matrix\_alloc(size1, size2)}
  \item \code{gsl\_matrix\_set\_zero(matrix)}
  \item \code{gsl\_matrix\_set(matrix, row, column, value)}
  \item \code{gsl\_matrix\_get(matrix, row, column)}
  \item \code{gsl\_vector\_alloc(size)}
  \item \code{gsl\_vector\_set\_zero(vector)}
  \item \code{gsl\_vector\_set(vector, index, value)}
  \item \code{gsl\_vector\_get(vector, index)}
  \item \code{gsl\_permutation\_alloc(size)}
  \item \code{gsl\_matrix\_memcpy(matrixToCopyFrom, matrix)}
\end{itemize}
In order to factorize a matrix into the LU decomposition, and then solve the square system $Ax=y$ using the decomposition of A, I've used the following methods:
\begin{itemize}
  \item \code{gsl\_linalg\_LU\_decomp(A, permutation, signum)}
  \item \code{gsl\_linalg\_LU\_solve(LU, permutation, b, x)}
\end{itemize}
  
\section{Solving the linear system}
In order to solve the system $Ax=y$, I first need to build the matrix A by understanding how it's build. The requirements are to build a tridiagonal matrix with the values $-1$ on the adjacent upper diagonal, the entries $+1$ on the adjacent lower diagonal, and on the main diagonal the values $b_{i}$, with $i = 1, \ldots, n$ given by

$$b_i =\frac{2(i+1)}{3},\quad i + 1= 3, 6, 9,\ldots$$
$$b_i =1,\quad i + 1 = 2, 4, 5, 7, 8, \ldots$$

By looking closely at the first rule, we see that the $i+1$ are all multiples of 3 ($i+1 = 3*k$, for some $k$). Hence the $i$ are of the form $i = 3*k-1$, for some $k$. For $n = 5$, for example, this is what the matrix looks like:
$$
\begin{bmatrix}
1.0000000000 & -1.0000000000 & 0.0000000000 & 0.0000000000 & 0.0000000000 \\
1.0000000000 & 2.0000000000 & -1.0000000000 & 0.0000000000 & 0.0000000000 \\
0.0000000000 & 1.0000000000 & 1.0000000000 & -1.0000000000 & 0.0000000000 \\ 
0.0000000000 & 0.0000000000 & 1.0000000000 & 1.0000000000 & -1.0000000000 \\ 
0.0000000000 & 0.0000000000 & 0.0000000000 & 1.0000000000 & 4.0000000000 \\\end{bmatrix}
$$

The coefficients matrix A is first allocated by using the \code{gsl\_matrix\_alloc} method, then I set all the elements to zero with \code{gsl\_matrix\_set\_zero} and finally nested \code{for} loops fill the diagonal values by checking the indexes. The coefficients reported above on the diagonal have 5 significant digits for improve the readability of this report.\\

I used the \code{gsl\_vector\_alloc} method to create an instance of the vector. All of its elements were set to zero by using \code{gsl\_vector\_set\_zero(vector)}. The exercise asks us to set the first element of the $y$ vector to one, so I used \code{gsl\_vector\_set(vector, 0, 1)} to assign the value 1 to index 0. For $n=5$, we have:
$$
\vec{y}=
\begin{bmatrix}
1.0000000000 \\
0.0000000000 \\
0.0000000000 \\
0.0000000000 \\
0.0000000000 \\
\end{bmatrix}
$$

Given the $Ax=y$ system, my goal is now to find the vector of the unknowns $x$. To do so, I first factorize $A$ into its LU decomposition by allocating a new matrix (so that the matrix which represents $A$ doesn't get overridden) using \code{gsl\_matrix\_memcpy} and then by calling \code{gsl\_linalg\_LU\_decomp}. This method utilizes Gaussian Elimination with partial pivoting to compute the decomposition. The following is the $LU$ matrix for $n=5$:
$$
LU=
\begin{bmatrix} 
1.0000000000 & -1.0000000000 & 0.0000000000 & 0.0000000000 & 0.0000000000\\ 
1.0000000000 & 3.0000000000 & -1.0000000000 & 0.0000000000 & 0.0000000000\\ 
0.0000000000 & 0.3333333333 & 1.3333333333 & -1.0000000000 & 0.0000000000\\ 
0.0000000000 & 0.0000000000 & 0.7500000000 & 1.7500000000 & -1.0000000000\\ 
0.0000000000 & 0.0000000000 & 0.0000000000 & 0.5714285714 & 4.5714285714\\
\end{bmatrix}
$$

I can now use the $LU$ matrix to solve the system by passing $LU$, $x$, a permutation structure \code{gsl\_permutation} and $y$ to \code{gsl\_linalg\_LU\_solve}. This method uses forward and back-substitution to modify the contents of the $x$ vector given in input, which now looks like this (for $n=5$):

$$
\vec{x}=
\begin{bmatrix}
0.7187500000\\
-0.2812500000\\
0.1562500000\\
-0.1250000000\\
0.0312500000\\
\end{bmatrix}
$$

The first element looks contains an approximation of $e-2$. By increasing the size of the matrix $n$, $x_i$ becomes increasingly more precise. For $n=10$, for example:

$$
\vec{x}=
\begin{bmatrix}
0.7182817183\\
-0.2817182817\\
0.1548451548\\
-0.1268731269\\
0.0279720280\\
-0.0149850150\\
0.0129870130\\
-0.0019980020\\
0.0009990010\\
-0.0009990010\\
\end{bmatrix}
$$

For $n=20$:

$$
\vec{y}=
\begin{bmatrix}
0.7182818285\\
-0.2817181715\\
0.1548454854\\
-0.1268726862\\
0.0279727992\\
-0.0149814893\\
0.0129913099\\
-0.0019901794\\
0.0010502335\\
-0.0009399460\\
0.0001102875\\
-0.0000576459\\
0.0000526416\\
-0.0000050043\\
0.0000025983\\
-0.0000024061\\
0.0000001922\\
-0.0000000994\\
0.0000000928\\
-0.0000000066\\
\end{bmatrix}
$$

Iterative refinement, a technique used to improve the approximate solution $\vec{x}$ of our system, could be helpful to improve these results. GSL

\section{Observations}

\end{document}