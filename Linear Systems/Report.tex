\documentclass{article}

\usepackage{lipsum}
\usepackage[margin=1.2in]{geometry}
\usepackage{titlesec}
\usepackage{graphicx}
\usepackage{amsmath}

\newcommand{\code}{\texttt}
\newcommand{\norm}[1]{\left\lVert#1\right\rVert}

\usepackage{siunitx} % Required for alignment

\sisetup{
  round-mode          = places, % Rounds numbers
  round-precision     = 2, % to 2 places
}

% Specify images directory
\graphicspath{ {./report-images/} }

% Header and Footer stuff
\usepackage{fancyhdr}
\pagestyle{fancy}
\fancyhead{}
\fancyfoot{}
\fancyfoot[R]{ \thepage\ }
\renewcommand{\headrulewidth}{0pt}
\renewcommand{\footrulewidth}{0pt}
\newcommand{\sectionbreak}{\clearpage}
\setlength{\parindent}{0pt}

%

\begin{document}

%----------------------------------------------------------------------------------------
%	TITLE PAGE
%----------------------------------------------------------------------------------------

\begin{titlepage} % Suppresses displaying the page number on the title page and the subsequent page counts as page 1
	\newcommand{\HRule}{\rule{\linewidth}{0.5mm}}% Defines a new command for horizontal lines, change thickness here
	
	\center % Centre everything on the page
	
	%------------------------------------------------
	%	Headings
	%------------------------------------------------
	
	\textsc{\Large Systems of Linear Equations}\\[0.5cm] % Major heading such as course name
	
	\textsc{\large Exercise 7}\\[0.5cm] % Minor heading such as course title
	
	%------------------------------------------------
	%	Title
	%------------------------------------------------
	
	\HRule\\[0.6cm]
	
	{\huge\bfseries Solving a Linear System with LU Decomposition}\\[0.25cm] % Title of your document
	
	\HRule\\[1.5cm]
	
	%------------------------------------------------
	%	Author(s)
	%------------------------------------------------
	
	\begin{minipage}{0.4\textwidth}
		\begin{flushleft}
			\large
			\textit{Author}\\
			\textsc{Cesare De Cal} % Your name
		\end{flushleft}
	\end{minipage}
	~
	\begin{minipage}{0.4\textwidth}
		\begin{flushright}
			\large
			\textit{Professor}\\
			\textsc{Annie Cuyt}\\ % Supervisor's name
			[0.25cm]
			\textit{Assistant Professor}\\
			\textsc{Ferre Knaepkens} % Supervisor's name

		\end{flushright}
	\end{minipage}
		
	\vfill\vfill\vfill
	
	{\large\today}
		
	\vfill
	
\end{titlepage}

%----------------------------------------------------------------------------------------

\section{Introduction}\label{sec:intro}
This exercise asks to build a tridiagonal matrix using the following rules with the values $-1$ on the adjacent upper diagonal, the entries $+1$ on the adjacent lower diagonal, and the values $b_{i}$, with $i = 1, \ldots, n$ given by

$$b_i =\frac{2(i+1)}{3},\quad i + 1= 3, 6, 9,\ldots$$
$$b_i =1,\quad i + 1 = 2, 4, 5, 7, 8, \ldots$$

on the main diagonal. This matrix should then be used as the coefficients matrix in the $A\vec{x}=\vec{y}$ linear system. The exercise asks to solve the system using \code{GEPP} (Gaussian Elimination with Partial Pivoting) and then give $x_1$, which should be an approximation of the $e-2$ value.\\

As we've seen in class, there are multiple ways of solving a linear system $AX=B$. Assume $A$ is a $n\times n$ square matrix, $B$ is a "constant" term matrix $n\times h$, and $X$ is a $n\times h$ unknown matrix. To solve for $X$, we could compute the inverse of $A$ and find $x=A^{-1}y$. We've seen that this approach, however, requires more computations than necessary and returns a less accurate result.\\

In this exercise I am going to use solve a linear system using LU decomposition. This technique, used to represent the matrix A in the form of simpler matrices, $L$ and $U$ (lower triangular and upper triangular matrices, respectively), uses forward substitution (solving for $Y$ from $LY=B$) and backward substitution (solving for $X$ from $UX = Y$). As seen in class, this method is numerically stable (as in, there will be no extra truncation errors). I'll also be calculating the condition number and the error.

\section{Tools}
The following programming language and libraries have been used in this exercise:
\begin{itemize}
  \item C
  \item GSL (GNU Scientific Library)
\end{itemize}
The following double-precision GSL data types have been used in the exercise:
\begin{itemize}
  \item \code{gsl\_vector}
  \item \code{gsl\_matrix}
  \item \code{gsl\_permutation}
\end{itemize}
The following GSL methods have been used in the exercise:
\begin{itemize}
  \item \code{gsl\_matrix\_alloc(size1, size2)}
  \item \code{gsl\_matrix\_set\_zero(matrix)}
  \item \code{gsl\_matrix\_set(matrix, row, column, value)}
  \item \code{gsl\_matrix\_get(matrix, row, column)}
  \item \code{gsl\_vector\_alloc(size)}
  \item \code{gsl\_vector\_set\_zero(vector)}
  \item \code{gsl\_vector\_set(vector, index, value)}
  \item \code{gsl\_vector\_get(vector, index)}
  \item \code{gsl\_matrix\_memcpy(matrixToCopyFrom, matrix)}
  \item \code{gsl\_linalg\_SV\_decomp(A, V, S, workspaceVector)}
  \item \code{gsl\_vector\_minmax(vector, minInVector, maxInVector)}
\end{itemize}
In order to factorize a matrix into the LU decomposition, and then solve the square system $Ax=y$ using the decomposition of A, I've used the following methods:
\begin{itemize}
  \item \code{gsl\_linalg\_LU\_decomp(A, permutation, signum)}
  \item \code{gsl\_linalg\_LU\_solve(LU, permutation, b, x)}
  \item \code{gsl\_permutation\_alloc(size)}
\end{itemize}
  
\section{Solving the Linear System}
By looking closely at the first rule, we see that the $i+1$ are all multiples of 3 ($i+1 = 3*k$, for some $k$). Hence the $i$ are of the form $i = 3*k-1$, for some $k$. For $n = 5$, for example, this is what the coefficient matrix looks like:
$$
\begin{bmatrix}
1.000000000e+00 & -1.000000000e+00 & 0.000000000e+00 & 0.000000000e+00 & 0.000000000e+00 \\ 
1.000000000e+00 & 2.000000000e+00 & -1.000000000e+00 & 0.000000000e+00 & 0.000000000e+00 \\ 
0.000000000e+00 & 1.000000000e+00 & 1.000000000e+00 & -1.000000000e+00 & 0.000000000e+00 \\ 
0.000000000e+00 & 0.000000000e+00 & 1.000000000e+00 & 1.000000000e+00 & -1.000000000e+00 \\ 
0.000000000e+00 & 0.000000000e+00 & 0.000000000e+00 & 1.000000000e+00 & 4.000000000e+00 \\
\end{bmatrix}
$$

The coefficients matrix A is first allocated by using the \code{gsl\_matrix\_alloc} method, then I set all the elements to zero with \code{gsl\_matrix\_set\_zero} and finally nested \code{for} loops fill the diagonal values by checking the indexes. The coefficients reported above on the diagonal have 5 significant digits for improve the readability of this report.\\

I used the \code{gsl\_vector\_alloc} method to create an instance of the vector. All of its elements were set to zero by using \code{gsl\_vector\_set\_zero(vector)}. The exercise asks us to set the first element of the $y$ vector to one, so I used \code{gsl\_vector\_set(vector, 0, 1)} to assign the value 1 to index 0. For $n=5$, we have:
$$
\vec{y}=
\begin{bmatrix}
1.000000000e+00 \\
0.000000000e+00 \\
0.000000000e+00 \\
0.000000000e+00 \\
0.000000000e+00 \\
\end{bmatrix}
$$

Given the $Ax=y$ system, my goal is now to find the vector of the unknowns $x$. To do so, I first factorize $A$ into its LU decomposition by allocating a new matrix (so that the matrix which represents $A$ doesn't get overridden) using \code{gsl\_matrix\_memcpy} and then by calling \code{gsl\_linalg\_LU\_decomp}. This method utilizes Gaussian Elimination with partial pivoting to compute the decomposition. The following is the $LU$ matrix for $n=5$:
$$
\begin{bmatrix} 
1.000000000e+00 & -1.000000000e+00 & 0.000000000e+00 & 0.000000000e+00 & 0.000000000e+00 \\ 
1.000000000e+00 & 3.000000000e+00 & -1.000000000e+00 & 0.000000000e+00 & 0.000000000e+00 \\ 
0.000000000e+00 & 3.333333333e-01 & 1.333333333e+00 & -1.000000000e+00 & 0.000000000e+00 \\
0.000000000e+00 & 0.000000000e+00 & 7.500000000e-01 & 1.750000000e+00 & -1.000000000e+00 \\ 
0.000000000e+00 & 0.000000000e+00 & 0.000000000e+00 & 5.714285714e-01 & 4.571428571e+00 \\
\end{bmatrix}
$$

I can now use the $LU$ matrix to solve the system by passing $LU$, $x$, a permutation structure \code{gsl\_permutation} (it contains the order of the indexes of the equations in the system to keep track of swapping) and $y$ to \code{gsl\_linalg\_LU\_solve}. This method uses forward and back-substitution to modify the contents of the $x$ vector given in input, which now looks like this (for $n=5$):

$$
\vec{x}=
\begin{bmatrix}
7.187500000e-01\\
-2.812500000e-01\\
1.562500000e-01\\
-1.250000000e-01\\
3.125000000e-02\\
\end{bmatrix}
$$

Then, I calculate the condition number of the matrix $A$ of order $n$ which will give me a better idea if this is a well-conditioned or an ill-conditioned linear system. In GSL there is no direct function that calculates the condition number, but it's possible to use the ratio of the largest singular value of matrix A, $\sigma_n (A)$, to the smallest $\sigma_1 (A)$:

$$\kappa(A) := \frac{\sigma_n (A)}{\sigma_1 (A)}= \frac{\norm{A}}{\norm{A^{-1}}^{-1}}$$

I proceed to factorize $A$ into its singular value decomposition $SVD$ using the \code{gsl\_linalg\_SV\_decomp} method, and then use $\code{gsl\_vector\_minmax}$ to extract the minimum and maximum singular values out of the vector $S$ that contains the diagonal elements of the singular value matrix. \\

For $n=5$, the condition number is

$$\kappa(A) = \frac{\sigma_n (A)}{\sigma_1 (A)}= \frac{4.205100611e+00}{1.142643287e+00}=3.680151678e+00$$

I calculate the error by subtracting the computed solution $x_{1}^{\ast}$ from the exact mathematical solution $\widetilde{x}$ (which can be obtained by using the $\code{M\_E}$ GSL constant minus 2).\\

\begin{table*}[htb]
\centering % used for centering table
\begin{tabular}{c c c c} % centered columns (4 columns)
$n$ & $\widetilde{x}_1$ & $x_1^{\ast}- \widetilde{x}_1$ & $\kappa(A_n)$ \\ [0.65ex] % inserts table
%heading
\hline % inserts single horizontal line
1 & 1.000000000e+00 & -2.817181715e-01 & 1.000000000e+00 \\
2 & 6.666666667e-01 & 5.161516179e-02 & 1.767591879e+00 \\
3 & 7.500000000e-01 & -3.171817154e-02 & 2.561552813e+00 \\
4 & 7.142857143e-01 & 3.996114173e-03 & 2.258696038e+00 \\
5 & 7.187500000e-01 & -4.681715410e-04 & 3.680151678e+00 \\
6 & 7.179487179e-01 & 3.331105103e-04 & 3.953864002e+00 \\
7 & 7.183098592e-01 & -2.803069588e-05 & 3.847674609e+00 \\
8 & 7.182795699e-01 & 2.258566572e-06 & 5.377037588e+00 \\
9 & 7.182835821e-01 & -1.753630507e-06 & 5.727581839e+00 \\
10 & 7.182817183e-01 & 1.101773268e-07 & 5.498872833e+00 \\
11 & 7.182818352e-01 & -6.746947445e-09 & 7.100335770e+00 \\
12 & 7.182818229e-01 & 5.515095380e-09 & 7.582164638e+00 \\
13 & 7.182818287e-01 & -2.766507023e-10 & 7.195531702e+00 \\
14 & 7.182818284e-01 & 1.364375279e-11 & 8.833149892e+00 \\
15 & 7.182818285e-01 & -1.153854789e-11 & 9.488074730e+00 \\
16 & 7.182818285e-01 & 4.816147481e-13 & 8.911558696e+00 \\
17 & 7.182818285e-01 & -1.998401444e-14 & 1.057152285e+01 \\
18 & 7.182818285e-01 & 1.709743458e-14 & 1.142018246e+01 \\
19 & 7.182818285e-01 & -6.661338148e-16 & 1.063813407e+01 \\
20 & 7.182818285e-01 & -1.110223025e-16 & 1.231319966e+01 \\
21 & 7.182818285e-01 & -2.220446049e-16 & 1.336883104e+01 \\
22 & 7.182818285e-01 & -2.220446049e-16 & 1.237107821e+01 \\
23 & 7.182818285e-01 & -2.220446049e-16 & 1.405700479e+01 \\
24 & 7.182818285e-01 & -2.220446049e-16 & 1.532862983e+01 \\
25 & 7.182818285e-01 & -2.220446049e-16 & 1.410816377e+01 \\
26 & 7.182818285e-01 & -2.220446049e-16 & 1.580226249e+01 \\
27 & 7.182818285e-01 & -2.220446049e-16 & 1.729630706e+01 \\
28 & 7.182818285e-01 & -2.220446049e-16 & 1.584809348e+01 \\
29 & 7.182818285e-01 & -2.220446049e-16 & 1.754855617e+01 \\
30 & 7.182818285e-01 & -2.220446049e-16 & 1.926975724e+01 \\
31 & 7.182818285e-01 & -2.220446049e-16 & 1.759006043e+01 \\
32 & 7.182818285e-01 & -2.220446049e-16 & 1.929561485e+01 \\
33 & 7.182818285e-01 & -2.220446049e-16 & 2.124756325e+01 \\
34 & 7.182818285e-01 & -2.220446049e-16 & 1.933353645e+01 \\
35 & 7.182818285e-01 & -2.220446049e-16 & 2.104325456e+01 \\
36 & 7.182818285e-01 & -2.220446049e-16 & 2.322873622e+01 \\
37 & 7.182818285e-01 & -2.220446049e-16 & 2.107816128e+01 \\
38 & 7.182818285e-01 & -2.220446049e-16 & 2.279134599e+01 \\
39 & 7.182818285e-01 & -2.220446049e-16 & 2.521256520e+01 \\
40 & 7.182818285e-01 & -2.220446049e-16 & 2.282368084e+01 \\
41 & 7.182818285e-01 & -2.220446049e-16 & 2.453979556e+01 \\
42 & 7.182818285e-01 & -2.220446049e-16 & 2.719852600e+01 \\
43 & 7.182818285e-01 & -2.220446049e-16 & 2.456991077e+01 \\
44 & 7.182818285e-01 & -2.220446049e-16 & 2.628853390e+01 \\
45 & 7.182818285e-01 & -2.220446049e-16 & 2.918622370e+01 \\
46 & 7.182818285e-01 & -2.220446049e-16 & 2.631671410e+01 \\
47 & 7.182818285e-01 & -2.220446049e-16 & 2.803750846e+01 \\
48 & 7.182818285e-01 & -2.220446049e-16 & 3.117535515e+01 \\
49 & 7.182818285e-01 & -2.220446049e-16 & 2.806398692e+01 \\
50 & 7.182818285e-01 & -2.220446049e-16 & 2.978667872e+01 \\\hline %inserts single line
\end{tabular}
\end{table*}

\section{Plot}
\includegraphics[width=\textwidth,height=\textheight,keepaspectratio]{cond_number.png}

\section{Observations}
The linear system presented in this exercise gets increasingly ill-conditioned as $n$ grows (since $\kappa(A_n)> 1$ for most $n$). From the plot, it can be observed that the condition number grows linearly. It can be noticed, however, that a large condition number doesn’t necessarily mean that the error will be large in all cases, just that it is possible to have a large error. However, it can be observed that as $n$ increases, the error gets incrementally smaller.\\

The error that I have calculated represents how well the computed solution $\widetilde{x}_1$ approximates the true solution $x_1^{\ast}$. It can be noted that the Gaussian elimination with partial pivoting doesn't introduce any additional truncation errors and therefore it is numerically stable.

\end{document}