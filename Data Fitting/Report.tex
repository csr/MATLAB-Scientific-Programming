\documentclass{article}

\usepackage{lipsum}
\usepackage[margin=1.20in, includefoot]{geometry}
\usepackage{titlesec}
\usepackage{graphicx}

\newcommand{\code}{\texttt}

% Specify images directory
\graphicspath{ {./report-images/} }

% Header and Footer stuff
\usepackage{fancyhdr}
\pagestyle{fancy}
\fancyhead{}
\fancyfoot{}
\fancyfoot[R]{ \thepage\ }
\renewcommand{\headrulewidth}{0pt}
\renewcommand{\footrulewidth}{0pt}
\newcommand{\sectionbreak}{\clearpage}
\setlength{\parindent}{0pt}

%

\begin{document}
\begin{titlepage}
	\begin{center}
	\line(1,0){300}\\
	[0.25in]
	\huge{\bfseries Data Fitting Report}  \\
	\line(1,0){200} \\
	[1.5cm]
	\Large{Exercise 3} \\
	[0.25cm]
	\Large{Comparison between interpolating polynomial and natural cubic spline} \\
	[12cm]
	\end{center}
	\begin{flushright}
	\large{Cesare De Cal \\
	[0.25cm]
	Professor: Annie Cuyt \\
	[0.25cm]
	Assistant Professor: Ferre Knaepkens \\
	}
	\end{flushright}
\end{titlepage}

\section{Introduction}\label{sec:intro}
An imaginary chemistry experiment produces the following data set:

  \begin{table}[!ht]
    \large        %% not "\fontsize{12}{12}\selectfont"
    \centering    %% not "\center{...}"
    \begin{tabular}{|c|c|c|c|c|c|c|c|}
    \hline
    \it{x}\textsubscript{i}&-1&-0.96&-0.86&-0.79&0.22&0.50&0.93\\     %% no "&" at start of row
    \it{f}\textsubscript{i}&-1.000&-0.151&0.894&0.986&0.895&0.500&-0.306\\
    \hline        %% extra \hline at bottom of table
    \end{tabular}
  \end{table}

The goal of this exercise is to use these data points to compute and plot the interpolating polynomial together with the natural cubic spline, and to report what it is observed. In the following sections I am going to describe the computation process and list the tools I used. I will then plot the interpolating polynomial and the natural cubic spine, and draw conclusions based on the plot.

\section{Tools}
The following programming language and libraries have been used in this exercise:
\begin{itemize}
  \item Python 3.7
  \item SciPy
\end{itemize}
The SciPy \code{interpolate} sub-package was used to compute the interpolating polynomial and the natural cubic spline:
\begin{itemize}
  \item \code{scipy.interpolate.lagrange(x, y)}
  \item \code{scipy.interpolate.CubicSpline(x, y, type)}
\end{itemize}

The following NumPy methods of the SciPy environment have been used in this exercise:
\begin{itemize}
  \item \code{numpy.array(object)}
  \item \code{numpy.linspace(start, stop, num)}
  \item \code{numpy.polynomial.polynomial.Polynomial(poly)}
  \end{itemize}
The following Matplotlib methods of the SciPy environment have been used in this exercise to plot:
 \begin{itemize}
  \item \code{matplotlib.pyplot.plot(x, y, formatting, label)}
  \item \code{matplotlib.pyplot.legend()}
  \item \code{matplotlib.pyplot.show()}
  \end{itemize}
  
\section{Computing the interpolating polynomial}
The exercise asks to compute the interpolating polynomial of the given data set. To do so, I first create two arrays in Python containing the data points using \code{np.array} and a linear space from -1 to 1 containing 1000 points. These values are finally passed to the \code{lagrange} method which returns the {\it Lagrange} interpolating polynomial. \\

The polynomial in the {\it power} form is: \\

$0.04365005085x^6 + 16.0766955610565x^5 - 0.048402197837630x^4 - 20.10047681678335740x^3\\ + 0.0168806991536721320x^2 + 5.029463735952404423200x - 0.006446071786954468800$ \\

The following are coefficients the interpolating polynomial:
 \begin{itemize}
  \item 0.04365005085
  \item 16.0766955610565
  \item -0.048402197837630
  \item -20.10047681678335740
  \item 0.0168806991536721320
  \item 5.029463735952404423200
  \item -0.006446071786954468800
  \end{itemize}

Here's a graph showing the 6th degree interpolating polynomial:

\includegraphics{intpoly}

I then proceeded to calculate the relative and absolute errors.

\section{Natural Cubic Spline}
In order to calculate the natural cubic spline...

\section{Observations}
The natural cubic spline is a much more accurate representation than the Lagrange function which appears inconsistent.  

\end{document}