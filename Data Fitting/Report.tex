\documentclass{article}

\usepackage{lipsum}
\usepackage[margin=1.20in,, includefoot]{geometry}

% Header and Footer stuff
\usepackage{fancyhdr}
\pagestyle{fancy}
\fancyhead{}
\fancyfoot{}
\fancyfoot[R]{ \thepage\ }
\renewcommand{\headrulewidth}{0pt}
\renewcommand{\footrulewidth}{0pt}

%

\begin{document}
\begin{titlepage}
	\begin{center}
	\line(1,0){300}\\
	[0.25in]
	\huge{\bfseries Data Fitting Report}  \\
	[2mm]
	\line(1,0){200} \\
	[1.5cm]
	\LARGE{Exercise 3} \\
	[0.25cm]
	\Large{Comparison between interpolating polynomial and natural cubic spline} \\
	[12cm]
	\end{center}
	
	\begin{flushright}
	\large{Cesare De Cal \\
	[0.25cm]
	Professor: Annie Cuyt \\
	[0.25cm]
	Assistant Professor: Ferre Knaepkens \\
	}
	\end{flushright}
\end{titlepage}

\section{Introduction}\label{sec:intro}
The goal of this exercise is to compute and plot the interpolating polynomial and the natural cubic spline of a set of data points provided by an imaginary chemistry experiment, and to draw conclusion on it. Here are the data points:

  \begin{table}[!ht]
    \large        %% not "\fontsize{12}{12}\selectfont"
    \centering    %% not "\center{...}"
    \begin{tabular}{|c|c|c|c|c|c|c|c|}
    \hline
    \it{x}\textsubscript{i}&-1&-0.96&-0.86&-0.79&0.22&0.50&0.93\\     %% no "&" at start of row
    \it{f}\textsubscript{i}&-1.000&-0.151&0.894&0.986&0.895&0.500&-0.306\\
    \hline        %% extra \hline at bottom of table
    \end{tabular}
  \end{table}
To compute the interpolating polynomial, I am going to use the Lagrange method saw in class.


  
\section{Tools}
The following programming language and libraries have been used in this exercise:
\begin{itemize}
  \item Python 3.7
  \item SciPy
\end{itemize}
The following NumPy methods of the SciPy environment have been used in this exercise:
\begin{itemize}
  \item numpy.array
  \item numpy.linspace
  \item numpy.polynomial.polynomial
  \end{itemize}
The following Matplotlib methods of the SciPy environment have been used in this exercise:
 \begin{itemize}
  \item matplotlib.pyplot.plot
  \item matplotlib.pyplot.legend
  \item matplotlib.pyplot.show
  \end{itemize}
\end{document}